%% This is file `elsarticle-template-1-num.tex',
%%
%% Copyright 2009 Elsevier Ltd
%%
%% This file is part of the 'Elsarticle Bundle'.
%% ---------------------------------------------
%%
%% It may be distributed under the conditions of the LaTeX Project Public
%% License, either version 1.2 of this license or (at your option) any
%% later version.  The latest version of this license is in
%%    http://www.latex-project.org/lppl.txt
%% and version 1.2 or later is part of all distributions of LaTeX
%% version 1999/12/01 or later.
%%
%% The list of all files belonging to the 'Elsarticle Bundle' is
%% given in the file `manifest.txt'.
%%
%% Template article for Elsevier's document class `elsarticle'
%% with numbered style bibliographic references
%%
%% $Id: elsarticle-template-1-num.tex 149 2009-10-08 05:01:15Z rishi $
%% $URL: http://lenova.river-valley.com/svn/elsbst/trunk/elsarticle-template-1-num.tex $
%%
\documentclass[preprint,12pt]{elsarticle}

%% Use the option review to obtain double line spacing
%% \documentclass[preprint,review,12pt]{elsarticle}

%% Use the options 1p,twocolumn; 3p; 3p,twocolumn; 5p; or 5p,twocolumn
%% for a journal layout:
%% \documentclass[final,1p,times]{elsarticle}
%% \documentclass[final,1p,times,twocolumn]{elsarticle}
%% \documentclass[final,3p,times]{elsarticle}
%% \documentclass[final,3p,times,twocolumn]{elsarticle}
%% \documentclass[final,5p,times]{elsarticle}
%% \documentclass[final,5p,times,twocolumn]{elsarticle}

%% if you use PostScript figures in your article
%% use the graphics package for simple commands
%% \usepackage{graphics}
%% or use the graphicx package for more complicated commands
%% \usepackage{graphicx}
%% or use the epsfig package if you prefer to use the old commands
%% \usepackage{epsfig}

%% The amssymb package provides various useful mathematical symbols
\usepackage{amssymb}
%% The amsthm package provides extended theorem environments
%% \usepackage{amsthm}

%% The lineno packages adds line numbers. Start line numbering with
%% \begin{linenumbers}, end it with \end{linenumbers}. Or switch it on
%% for the whole article with \linenumbers after \end{frontmatter}.
%% \usepackage{lineno}

%% natbib.sty is loaded by default. However, natbib options can be
%% provided with \biboptions{...} command. Following options are
%% valid:

%%   round  -  round parentheses are used (default)
%%   square -  square brackets are used   [option]
%%   curly  -  curly braces are used      {option}
%%   angle  -  angle brackets are used    <option>
%%   semicolon  -  multiple citations separated by semi-colon
%%   colon  - same as semicolon, an earlier confusion
%%   comma  -  separated by comma
%%   numbers-  selects numerical citations
%%   super  -  numerical citations as superscripts
%%   sort   -  sorts multiple citations according to order in ref. list
%%   sort&compress   -  like sort, but also compresses numerical citations
%%   compress - compresses without sorting
%%
%% \biboptions{comma,round}

% \biboptions{}


\journal{Science of Computer Programming}

\begin{document}

\begin{frontmatter}

%% Title, authors and addresses

%% use the tnoteref command within \title for footnotes;
%% use the tnotetext command for the associated footnote;
%% use the fnref command within \author or \address for footnotes;
%% use the fntext command for the associated footnote;
%% use the corref command within \author for corresponding author footnotes;
%% use the cortext command for the associated footnote;
%% use the ead command for the email address,
%% and the form \ead[url] for the home page:
%%
%% \title{Title\tnoteref{label1}}
%% \tnotetext[label1]{}
%% \author{Name\corref{cor1}\fnref{label2}}
%% \ead{email address}
%% \ead[url]{home page}
%% \fntext[label2]{}
%% \cortext[cor1]{}
%% \address{Address\fnref{label3}}
%% \fntext[label3]{}

\title{Softwarenaut: A Tool for Collaborative Architecture Recovery}

%% use optional labels to link authors explicitly to addresses:
%% \author[label1,label2]{<author name>}
%% \address[label1]{<address>}
%% \address[label2]{<address>}

\author{Mircea Lungu}

\address{Software Composition Group\\University of Bern\\Switzerland}

\begin{abstract}
Recovering the software architecture
\end{abstract}

\begin{keyword}
Architecture Recovery \sep
Software Visualization \sep
Software Tools \sep
Reverse Engineering \sep
\end{keyword}

\end{frontmatter}

%%
%% Start line numbering here if you want
%%

\section{Introduction}
\label{sec:Introduction}
Given that he understands the programming language in which a program 
is written, any developer will sooner or later understand a page of a 
program. He will read it and think about it and finally grasp its meaning.
When the program that needs to be understood is written by many people
over many years, it becomes hard for one to understand it just by reading
the code. He needs a way of abstracting and aggregating the many bits and
pieces of information that are scattered through the code. Architecture 
recovery tools are specialized in supporting program understanding by 
automating parts of the abstraction process. 


\section {The Problem}

Architecture is, conforming to Bass and Clements, "the structure or 
structures of a system that consists of components, connections between
them, and their properties". 

There is a clear consensus between the practitioners and researchers that
architecture is too complex a concept to be expressed all at once. Instead
the agreement is that architecture can only be expressed through multiple 
viewpoints. Each viewpoint presents the system from a given point of view.


\section {Modelling Software for Architecture Recovery Tools}

Softwarenaut was developed as part of the Moose analysis framework. Although
it evolved later, initially Moose was built around the FAMIX meta-model. FAMIX is 
a meta-model that supports the representation of systems written in various 
object-oriented languages. As a result, Softwarenaut takes as input the model
of a software system as it can be represented by FAMIX (specifically FAMIX 2.0). 

Figure xxx presents a diagram of FAMIX. 

The FAMIX model of a system is too detailed, especially when one is interested
in obtaining high-level architectural views of the system which present the 
high-level modules and their inter-dependencies. To reduce the amount
of information, one needs to aggregate and abstract the information. In the next
secion we present the way the information is aggregated in Softwarenaut. 

\section {Abstraction and Aggregation}

Hierarchical decomposition is a powerful tool for managing complexity. Softwarenaut
uses it to compress the amount of information that represents the software 
system under analysis and is presented to the user. 

The observation that stands at the basis of the aggregation in Softwarenaut is that
the software is organized hierarchically. Sometimes the hierarchy is put in place 
by the developers and sometimes it can be deduced through analysis. Examples of 
hierarchies in software are the following: 

\begin{itemize}
\item The directory structure
\item The package structure. This in Java is the same as the directory structure.
\item Dendrograms that are the result of hierarchical clustering. 
\end{itemize}



\section {Interactive Exploration}
\subsection {Exploration Operations}
The dominant exploration mechanism of Softwarenaut is refinement. Normally 
one starts with a very high-level view of a system that he continuously refines
by using exploration operations. Several of the interaction modes that Softwarenaut
supports are: 

\begin{itemize}
\item Expand. 
\item Collapse. 
\item Group. I might have to forget this... since it is a destructive operation
on the higraph that can not be saved in the view for now... 
\item Filter element. 
\end{itemize}

\subsection {Understanding Individual Elements}
\subsection {Filtering}
\subsubsection {Individual Filters}
\subsubsection {Structural Filters}
\subsubsection {Evolutionary Filters}
\subsection {Saving Views}


\section {Collaboration}
\subsection {Sharing Views: The Global View Repository}
\subsection {Annotating Views}
\subsection {Exporting Views}


\section {The Architecture of Softwarenaut}

\subsection{Using Softwarenaut to analyze Softwarenaut itself}


\section {Discussion}

\subsection {Documentation}
\subsection {Testing the tool in an educational setting}
\subsection {Symbiotic Tools}
MARS is a plugin for softwarenaut which allows for generating suggestions...



\section {Acknowledgements}
I would like to thank Michele Lanza, my advisor at the University 
of Lugano, for supporting me while developing the core functionality 
of Softwarenaut during my time as a PhD student there. 

\begin{enumerate}
\setcounter{enumi}{-1}
\item {one}
\item {two}
\end{enumerate}























%% The Appendices part is started with the command \appendix;
%% appendix sections are then done as normal sections
%% \appendix

%% \section{}
%% \label{}

%% References
%%
%% Following citation commands can be used in the body text:
%% Usage of \cite is as follows:
%%   \cite{key}          ==>>  [#]
%%   \cite[chap. 2]{key} ==>>  [#, chap. 2]
%%   \citet{key}         ==>>  Author [#]

%% References with bibTeX database:

\bibliographystyle{model1-num-names}
\bibliography{<your-bib-database>}

%% Authors are advised to submit their bibtex database files. They are
%% requested to list a bibtex style file in the manuscript if they do
%% not want to use model1-num-names.bst.

%% References without bibTeX database:

% \begin{thebibliography}{00}

%% \bibitem must have the following form:
%%   \bibitem{key}...
%%

% \bibitem{}

% \end{thebibliography}


\end{document}

%%
%% End of file `elsarticle-template-1-num.tex'.
